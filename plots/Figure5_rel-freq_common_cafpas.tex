\documentclass{article}
%\input{Header.tex}
\usepackage[latin1]{inputenc}
\usepackage{floatflt}
\usepackage{float}
\usepackage{tikz}
\usepackage{verbatim}
%\usepackage[landscape]{geometry}
\usetikzlibrary{matrix}


\begin{document}
\thispagestyle{empty}
% plot style
\newcommand{\plotdir}{s5_results_I_wadd-0}
\newcommand{\plotstyle}{common_weights_cafpas_frequencies} 
\newcommand{\append}{_Y90_wadd-0}  

\newcommand{\modelA}{lasso}
\newcommand{\modelB}{elasticNet}
\newcommand{\modelC}{randomForest}
\newcommand{\modelD}{all-models}

\newcommand{\textAa}{}
\newcommand{\textAb}{Expert \& Lasso}
\newcommand{\textAc}{}
\newcommand{\textBa}{}
\newcommand{\textBb}{Expert \& Elastic Net}
\newcommand{\textBc}{}
\newcommand{\textCa}{}
\newcommand{\textCb}{Expert \& Random Forest}
\newcommand{\textCc}{}
\newcommand{\textDa}{}
\newcommand{\textDb}{Expert \& all models}
\newcommand{\textDc}{}


% comparisons of categories
%\newcommand{\compA}{nh-hi}
%\newcommand{\compB}{high-high+cond}
%\newcommand{\compC}{high-high+recr}
%\newcommand{\compD}{none-device}
%\newcommand{\compE}{HA-CI}
%\newcommand{\compL}{legend}
% ...
% measurements/subjective information - hier noch Reihenfolge �ndern 
%\newcommand{\measA}{ag-ac} % A
%\newcommand{\measB}{ag-bc} % C
%\newcommand{\measC}{ABG} % D 
%\newcommand{\measD}{cafpas} % B
%\newcommand{\measE}{acalos-1-5} % I 
%\newcommand{\measF}{acalos-4} % J 

%\newcommand{\textAa}{Audiogram}
%\newcommand{\textAb}{Air Conduction}
%\newcommand{\textAc}{(AG-AC)}
%\newcommand{\textBa}{Audiogram}
%\newcommand{\textBb}{Bone Conduction}
%\newcommand{\textBc}{(AG-BC)}
%\newcommand{\textCa}{Air-Bone Gap}
%\newcommand{\textCb}{(ABG)}
%\newcommand{\textCc}{}
%\newcommand{\textDa}{Common Audiological}
%\newcommand{\textDb}{Functional Parameters}
%\newcommand{\textDc}{(CAFPAs)}
%\newcommand{\textEa}{Adaptive Catego-} 
%\newcommand{\textEb}{rical Loudness}
%\newcommand{\textEc}{Scaling (1.5 kHz)}
%\newcommand{\textFa}{Adaptive Catego-} 
%\newcommand{\textFb}{rical Loudness}
%\newcommand{\textFc}{Scaling (4.0 kHz)}


% plot parameters
\newcommand{\xstart}{3} % 9
\newcommand{\deltax}{1.6} % 3.2 an sich, fuer weights 3.5
\newcommand{\deltaxII}{4}
\newcommand{\ystart}{17}
\newcommand{\deltay}{-4.3} %-6.5
\newcommand{\trimleft}{0.72} % 0.72 an sich, fuer weights 0.65
\newcommand{\trimleftacalos}{0.6}


\begin{tikzpicture}[remember picture, overlay] 

\tikzstyle{NS3}=[rectangle,draw=black,line width=0.4mm,text width=22mm, minimum width=30mm,align=center]
\tikzstyle{DP1}=[dash pattern=on 4pt off 2pt]
\tikzstyle{DP2}=[dash pattern=on 8pt off 2pt on 4pt off 1pt on 2pt off 1pt]
\tikzstyle{DP3}=[dash pattern=on 10pt off 2pt]
\tikzstyle{DP4}=[dash pattern=on 10pt off 2pt on 7 pt off 2pt]
\tikzstyle{DP5}=[dash pattern=on 12pt off 2pt on 5pt off 2pt]



% to do: 
% - Spalten 3-5 mit weiteren Vergleichen f�llen 
% - evtl a) b) c) etc. - Nummerierung f�r Paper
% - earnoise leicht seitlich im Vergleich zum Rest verschoben - fixen
% - Bilder in Matlab updaten - alle gleiche Eigenschaften 

% trim: lbrt
    \node [shift={(0 cm,-20 cm)}]  at (current page.north west) 
        {
        
\begin{tikzpicture}[remember picture, overlay] 
% Legenden Kategorienvergleich (Spaltentitel) - k�nnte man durch tex-Schrift ersetzen (Linien hier nicht ben�tigt)
%\foreach \i/\comp/\diag/\setno/\st in {1/\compA/findings/I/DP1,2/\compB/findings/II/DP2,3/\compC/findings/III/DP3,4/\compD/reha/IV/DP4,5/\compE/reha/V/DP5} 
%{
%\node[NS3,\st] (c1) at ({\xstart+(\i-1)*\deltax-0.04},\ystart+3.2){Comparison \\ set \setno }; %  \scriptsize (single)/joint 
%}


%\foreach \i/\comp/\diag in {1/\compA/findings,2/\compB/findings,3/\compC/findings,4/\compD/reha,5/\compE/reha} 
%{
%\node[inner sep=0pt] (russell) at ({\xstart+(\i-1)*\deltax+0.25},\ystart+2)
%{\includegraphics[trim={0.5cm 2.1cm 0 0},clip]{../../v1_2/3_Results_all_sup/legend/legend_\comp}}; % austauschen, da blau und rot in Paper 3 nicht mehr vorkommt
%}

%% Messungen (Zeilentitel) 
\foreach \j/\texta/\textb/\textc in {1/\textAa/\textAb/\textAc,2/\textBa/\textBb/\textBc,3/\textCa/\textCb/\textCc,4/\textDa/\textDb/\textDc}
{
\coordinate[label = center:\rotatebox{90}{\scriptsize \textbf{\texta}}] (L0) at (\xstart-2,{\ystart+0.1+(\j-1)*\deltay}); % \xstart-3.3
\coordinate[label = center:\rotatebox{90}{\scriptsize \textbf{\textb}}] (L0) at (\xstart-1.7,{\ystart+0.1+(\j-1)*\deltay}); % \xstart-3
\coordinate[label = center:\rotatebox{90}{\scriptsize \textbf{\textc}}] (L0) at (\xstart-1.4,{\ystart+0.1+(\j-1)*\deltay}); % \xstart-2.7
} 


%% Plots
% Schleife �ber Zeilen (gleichartige Bilder G�sa bis Age) - evtl. erweitern auf alle 
\foreach \j/\model in {1/\modelA,2/\modelB,3/\modelC,4/\modelD}
{

\foreach \i in {2} % 1. Spalte anders wegen trim/startx
{\node[inner sep=0pt,scale = 1.2] (russell) at ({\xstart-0.37+(\i-1)*\deltax},{\ystart-0.2+(\j-1)*\deltay}){\includegraphics[trim={0cm 0 0 0.2},clip]{./\plotdir/\plotstyle_\model\append}}; 
}
%\foreach \i/\comp in {2/\compB,3/\compC,4/\compD,5/\compE} % Zeile 
%{ \node[inner sep=0pt] (russell) at ({\xstart+(\i-1)*\deltax},{\ystart-0.2+(\j-1)*\deltay}){\includegraphics[trim={\trimleft cm 0 0 0},clip]{./p4-fig12/\plotstyle_\comp_\datachoice_\model}};    
%}
�}

%\foreach \j/\meas/\tl/\tb/\tr/\tt in {1/\measA/1.9/1.2/0.5/0.5,2/\measB/1.9/1.2/0.5/0.5,3/\measC/1.9/1.2/0.5/0.5,4/\measD/1.8/1.1/0.5/0.5,5/\measE/2/1.2/0.5/1.5,6/\measF/2/1.2/0.5/1.5} % Legenden individuell getrimmt (auch abh�ngig von \datachoice)
%{ \node[inner sep=0pt] (russell) at (19.5,{\ystart+(\j-1)*\deltay}){\includegraphics[trim={\tl cm \tb cm \tr cm \tt cm},clip]{../3_Results_all_sup/legend/\compL_\meas_\datachoice}};    
%}


% 1. Zeile manuell Kategorien (zwischen �berschrift Comparison set und Plots)
%\node[inner sep=0pt] (russell) at (4.2,17.2)
%{\includegraphics[trim={1.1cm 2.05cm 0 0cm},clip]{./p4-fig9/legend_tikz/legend_\compA}}; 
%\node[inner sep=0pt] (russell) at (7.5,17.2)
%{\includegraphics[trim={0.87cm 2.05cm 0 0cm},clip]{./p4-fig9/legend_tikz/legend_\compB}}; 
%\node[inner sep=0pt] (russell) at (10.8,17.2)
%{\includegraphics[trim={0.57cm 2.05cm 0 0cm},clip]{./p4-fig9/legend_tikz/legend_\compC}}; 
%\node[inner sep=0pt] (russell) at (13.85,17.2)
%{\includegraphics[trim={1.19cm 2.05cm 0 0cm},clip]{./p4-fig9/legend_tikz/legend_\compD}}; 
%\node[inner sep=0pt] (russell) at (17.05,17.2)
%{\includegraphics[trim={1.15cm 2.05cm 0 0cm},clip]{./p4-fig9/legend_tikz/legend_\compE}};



			
			\end{tikzpicture}
        };
        \end{tikzpicture}
        

\end{document}